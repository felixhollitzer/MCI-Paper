\section{Evaluating Large Language Models’ Capability to Launch Fully Automated Spear Phishing Campaigns: Validated on Human Subjects}

Eine der neusten Entwicklungen in Phishing ist die Nutzung von LLMs(Large Language Models). Diese können benutzt werden, um Phishing Emails zu generieren, aber auch um diese zu erkennen. Dabei können LLMs automatisiert, personalisierte Phishing Emails generieren, basierend auf Daten die über den Nutzer bekannt sind. In der Studie "Evaluating Large Language Models’ Capability to Launch Fully Automated Spear
Phishing Campaigns: Validated on Human Subjects" wurden LLM generierte Phishing Emails mit denen von menschlichen Experten verglichen. Dazu wurden 101 Teilnehmer in 4 Gruppen eingeteilt: Eine Kontroll Gruppe, die willkürlich ausgewählte ausgewählte Spam Emails bekommen haben. Eine Gruppe, die Phishing Emails, von menschlichen Experten erstellt bekommen haben. Eine Gruppe, die Phishing Emails bekommen haben, die komplett AI automatisiert generiert wurden. Und eine Gruppe, die Phishing Emails, generiert mit KI, mit Human-in-the-loop Interventionen. Dabei wurde ein AI gestütztes Tool zum automatisierten von phishing Kampagnen entwickelt. Das Tool sammelt Informationen über das Ziel, schreibt ein Profil des Ziels anhand der Online Informationen über das Ziel, generiert und versendet Phishing Emails und analysiert die Ergebnisse zur Selbstverbesserung. Die Gruppe, welche komplett automtiserte Ki Phishing Emails bekommen hat, hat diese von diesem Tool komplett automatisiert bekommen. Bei der Gruppe, welche KI generiert mit Human-In-the-loop Interventionen Emails bekommen hat, war bei dem gesamten Prozess ein Mensch anwesend, der eingreifen und änderungen vornehmen konnte. Die Emails für die Gruppe, die von menschlichen Experten generierte Phishing Emalis bekommen hat, wurden von menschlichen Experten erstellt, welche bewährte Verfahren aus dem phsihing und Überzeugen benutzt haben(V-Triad und Robert Cialdini’s Influence guidelines).
In der Studie wurde untersucht wie viele der Teilnehmer der einzelnen Gruppen auf einen Link in den Emails gedrückt haben. Hier ist anzumerken, dass nur 60 der 101 Teilnehmer überhaupt email Aktivität gezeigt haben(wurde definiert als das beanspruchen der Belohnung für die Studie, welche später über Email gesendet wurde). Aus der Kontroll Gruppe haben 12\% den Link gedrückt, 54\% aus der menschlichen Experte Gruppe, 54\% aus der komplett KI generierten Gruppe und 56\% aus der KI generierten Gruppe mit Human-in-the-loop. Die Studie hat außerdem die Wirtschaftlichkeit der 4 verschiedenen Gruppen untersucht. Dabei wurden die Kosten erfasst(menschliche rbeitszeit, Ki-Kosten). Dabei wurden unterschiedliche Szenarien berechnet. Es wurde ein durhschnittlocher Payoff von 136\$ benutzt. Es wurden drei unterschiedliche Konversationsraten zu Profit untersucht, in zwei szenarien: US Gehalt für die Opportuntätskosten der Zeit und Russisches Gehalt für die Opportuntätskosten der Zeit. In allen von den 6 Szenarien war die komplett KI Methode die profitabelste.
In der Studie wurde außerdem untersucht, wie gut verschieden KI Modelle die jeweiligen Phishing Emails erkennen konnten. In Figur TODO sind die Ergebnisse der Studie zu sehen. Die Studie kommt zu dem Ergebniss, dass Calude 3.5 Sonnet am besten performt beim erkennen von Phishing Emails. Wenn darauf gepromptet Verdacht zu erkennen, hat Claude Sonnet 3.5 als einziges Modell alle verdächtigen Emails erkannt und hatte dabei keine False Positives. Es wurden außerdem Calude Sonnet 3.5 und GPT-4o auf das erkennen von verdächtigen Email in einem größeren Datensatz untersucht. In dem Datensatz sind neben den Ki generierten Phishing Emails aus dieser Studie noch, verschieden KI generierte Phishing Emails von verschieden KI Modellen, ligitimen Emails, sowie andere Arten an Phishing emails. Die anderen Arten an Phishing Emails werden dabei in der Studie nicht genauer beschrieben. Claude 3.5 Sonnet hat dabei eine Erkennungsrate von 97,25\% ohne False Positives.

Die Studie kommt mit diesen Ergebnissen zu dem Schluss, dass in Zukunft AI Phishing Agenten gegen AI Detection Agenten antreten werden, da AI Phishing Agenten ser Kosten effizient sind und AI Detection Agenten eine gute Erkennungsrate erzeilen können. Sie kommt auch zu dem Schluss, dass Individuen digitale Informatonen, die nicht nutzvoll für die Individuen sind, aber nutzvoll für die Angreifer entfernt werden sollten. Die Studie kommt zu dem Ergebniss, dass Ki automatisierte Verfahren im Spear Phishing(personalisierte Phishing Attacken) gleich gut, wie menschliche Experten sind und aktuelle IT Sicherheitsvorkehrung zum Teil nicht ausreichend sind. Außerdem sind Sicherheitsvorkehrung der KI Modelle nicht gut genug, um zu verhindern, dass diese zum generieren von Phishing Emails genutzt werden.

Die Studie umfasst viele interessante Themen über Ki in Phishing, wie der Evaluation von KI basierten Phishing Attacken, den Vergleich von verschiedenen KI Modellen zum Erkennen von Phishing Attacken und die Ökonmischen von Ki basierten Phsihing Attacken. Die Ergebnisse der Studie, dass KI in Zukunft benutzt wird, um Phishing Kampagnen durchzuführen und Phishing Angirffe zu erkennen klingt plausibel. Die Studie wurde rausgesucht, da sie einen guten Überblick über die verschiedenen Bereiche im Phishing, in denen KI eingesetzt wird, gibt. Die Autoren der Studie haben im Jahr davor auch schon eine Studie zu dem Thema veröffentlicht(Devising and Detecting Phishing: large language models vs. Smaller
Human Models) und in dieser waren die KI generierten Phishing Emails noch nicht so gut, wie die von menschlichen Experten.

\section{Phishing Attacks in the Age of Generative Artificial Intelligence:
A Systematic Review of Human Factors}

In der Studie "Phishing Attacks in the Age of Generative Artificial Intelligence: A Systematic Review of Human Factors" wird ein Literaturrecherche durgeführt, um "verschidene menschliche Faktoren, die bei Phishing Attacken ausgenutzt werden zu untersuchen, möglicher Lösungen und Präventionsmaßnahmen sowie
der Komplexität, die durch Generative KI gesteuerte Phishing-Angriffe entsteht". Dabei ist das Ziel der Studie "die Forschungslücke zu schließen, indem es ein tieferes Verständnis der sich entwickelnden Landschaft von Phishing-Angriffen unter Anwendung von GenAI und den damit verbundenen Auswirkungen auf den Menschen vermittelt und damit einen Beitrag zum Wissensgebiet der Abwehr von Phishing-Angriffen durch die Schaffung sicherer digitaler Interaktionen leistet".

Die Studie stellt fest, dass es nur eine limiterte Anzahl an Referenzen zu Menschlichen Faktoren in Phishing Angriffen gibt. Die Studien, welche sie in ihrere Literaturrecherche gefunden haben wurden studiert und die Autoren haben die Faktoren, welche zu menschlichen Schwachstellen führen können in 3 Kategorien kategorisiert. Die erste Kategorie ist unzureichendes Training, die zweite Kategorie ist Voreingenommenheit und Vernachlässigung und die dritte Kategorie ist Äußerer Einfluss. TODO genauer erleutern.

In der Studie wird auch, anhand der Literaturrecherche, untersucht, wie Generative KI die Risiken und Raffinesse von Phishing Angriffen erhöht. TODO genauer erläutern

Es wird außerdem untersucht, was die effektivsten Menschen fokussierte und Technologischen Lösungen sind, um Phishing Angriffe abzumildern. TODO genauer erklären

Die Studie untersucht zudem Phishing Angriffe auf Unternehmen und Individuen in Australien. Sie stellt dabei fest, dass die Menge der Phishing Angriffen bei steigendem Alter der Individuen auch steigt und kommt zu dem Ergebniss, dass ältere Menschen anfälliger für Betrug und Phishing Angirffe sind. Hier wird in der Studie jedoch nur die gesamte Anzahl pro Altersgruppe betrachtet, es wäre aber auch interessant diese durch die Anzahl der Personen in den jeweiligen Alterskohorten zu teilen, um feststellen zu können, ob bestimmte Gruppen gezielt öffters Angegriffen werden.

Die Studie fasst außerdem die verschieden menschlichen Faktoren, welche in der Literaturrecherche gefunden wurden zusammen und vergleicht diese. Tabelle TODO zeigt dabei die Ergebnisse der Studie. Anhand von diesen Ergebnissen haben die Autoren eine ganzheitliche Sichtweise auf die möglichen menschlichen Faktoren und Eigenschaften entwickelt, die bei Phishing angriffen genutzt werden können. Diese ist in Grafik TODO zu sehen.

